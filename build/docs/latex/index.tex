\section*{C\+L\+Utils }

C\+L\+Utils offers utilities that help setup and manage an Open\+C\+L environment. It aims to allow rapid prototyping by hiding away all the boilerplate code necessary for establishing an Open\+C\+L environment.

This meant to be an exercise on how I would go about containing the Open\+C\+L objects, and developing an A\+P\+I to interface them all. But since I didn\textquotesingle{}t have a solid idea of what I needed, I started writing the code and ended up adapting the A\+P\+I around it. The result was an A\+P\+I that made use of indices to refer to Open\+C\+L objects which only makes it efficient to use under a known environment. Nonetheless, the library works pretty robustly. Now that I have a better understanding on the problem, I intent to come back at some point and reconfigure the whole thing.

If you would like something more appropriate and complete, take a look at \href{https://github.com/kylelutz/compute}{\tt boost.\+compute} or \href{http://developer.amd.com/tools-and-sdks/opencl-zone/amd-accelerated-parallel-processing-app-sdk/}{\tt A\+M\+D A\+P\+P S\+D\+K\textquotesingle{}s C\+L\+Util}.

~\newline
 \section*{A\+P\+I }



~\newline
~\newline
 The simplest case covered by the library is one where a context is created for the first platform returned by the Open\+C\+L A\+P\+I, and a command queue is created for the first device in that platform. Programs are built for all devices in the associated context. This case is covered automatically when a kernel filename is provided on the definition of a {\ttfamily C\+L\+Env} instance.

\begin{quote}
\#include $<$\hyperlink{CLUtils_8hpp}{C\+L\+Utils.\+hpp}$>$

using namespace clutils;

int main () ~\newline
 \{ ~\newline
 ~~~~ C\+L\+Env cl\+Env (\char`\"{}my\+Kernels.\+cl\char`\"{}); ~\newline
 ~~~~ cl\+::\+Context \&context (cl\+Env.\+get\+Context ()); ~\newline
 ~~~~ cl\+::\+Command\+Queue \&queue (cl\+Env.\+get\+Queue ()); ~\newline
 ~~~~ cl\+::\+Kernel \&kernel (cl\+Env.\+get\+Kernel (\char`\"{}my\+Kernel\char`\"{})); ~\newline
 ~~~~ cl\+::\+N\+D\+Range global (1024), local (256); ~\newline


~~~~ cl\+::\+Buffer d\+Buf (context, C\+L\+\_\+\+M\+E\+M\+\_\+\+R\+E\+A\+D\+\_\+\+W\+R\+I\+T\+E, 1024 $\ast$ sizeof (int)); ~\newline
 ~~~~ kernel.\+set\+Arg (0, d\+Buf);

~~~~ queue.\+enqueue\+N\+D\+Range\+Kernel (kernel, cl\+::\+Null\+Range, global, local);

~~~~ return 0; ~\newline
 \} \end{quote}


~\newline
 \section*{Build \& Install }

\begin{quote}
git clone \href{mailto:git@github.com}{\tt git@github.\+com}\+:p\+A\+Ign10/\+C\+L\+Utils.\+git ~\newline
 cd C\+L\+Utils

mkdir build ~\newline
 cd build

cmake .. ~\newline
 \# or, to build the tests too ~\newline
 \# cmake -\/\+D\+B\+U\+I\+L\+D\+\_\+\+T\+E\+S\+T\+S=O\+N ..

make

\# to run the example ~\newline
 ./bin/cl\+Utils\+\_\+vec\+Add

\# to install the library ~\newline
 sudo make install

\# to build the documentation ~\newline
 make doxygen ~\newline
 firefox docs/html/index.\+html\end{quote}
